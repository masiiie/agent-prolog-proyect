\documentclass{llncs}
\usepackage[utf8]{inputenc}
\usepackage{booktabs}
\usepackage{rotating}
\RequirePackage{graphicx}
\RequirePackage[spanish]{babel}
\RequirePackage[utf8]{inputenc}
\selectlanguage{spanish}
\usepackage{verbatim} 


\usepackage{listings}
\usepackage[hidelinks]{hyperref}
\usepackage{color}

\definecolor{dkgreen}{rgb}{0,0.6,0}
\definecolor{gray}{rgb}{0.5,0.5,0.5}
\definecolor{mauve}{rgb}{0.58,0,0.82}

\lstset{frame=tb,
  language=Prolog,
  aboveskip=3mm,
  belowskip=3mm,
  showstringspaces=false,
  columns=flexible,
  basicstyle={\small\ttfamily},
  numbers=none,
  numberstyle=\tiny\color{gray},
  keywordstyle=\color{blue},
  commentstyle=\color{dkgreen},
  stringstyle=\color{mauve},
  breaklines=true,
  breakatwhitespace=true,
  tabsize=3
}

\newcommand{\mykeywords}[1]{\par\addvspace\baselineskip \noindent \textbf{Palabras Claves:} \enspace\ignorespaces#1}

\newtheorem{teo}{Teorema}

\begin{document}

\title{Proyecto de Simulaci\'on de Agentes.\\ Programaci\'on Declarativa. Prolog.}

\author{
  Masiel Villalba Carmenate 
  \email{villalbamasiel@gmail.com}
  \institute{Facultad de Matem\'atica y Computaci\'on (MATCOM), \\Universidad de la Habana (UH), Cuba.}
  }

\titlerunning{Informe de Programaci\'on Declarativa y Simulaci\'on.} 
\authorrunning{Masiel Villalba Carmenate}


\maketitle

\section{Definici\'on del problema}

Dado un tablero donde en cada casilla puede aparecer alguno de los siguientes elementos:

\begin{itemize}
\item suciedad
\item ni\~nos
\item corrales
\item obst\'aculos
\end{itemize}

Existen las siguientes leyes:

\begin{itemize}
\item Los ni\~nos se mueven y ensucian arbitrariamente.
\item Las casillas de tipo obst\'aculo no pueden ser rebasadas por ning\'un agente.
\item Si un ni\~no est\'a en una casilla de tipo corral, entonces no puede moverse ni ensuciar.
\item Existe un intervalo de tiempo $t$ seg\'un el cual el tablero cambia aleatoriamente.
\end{itemize}


Se busca implementar un agente que consiga limpiar el tablero y ubicar a todos los ni\~nos en una casilla corral antes de que se rebase un umbral de suciedad predeterminado, en caso contrario se considera que queda despedido.


\section{Principales ideas}

Para la implementaci\'on del robot se modelaron dos comportamientos: reactivo y proactivo. 

\subsection{Agente proactivo}
El robot proactivo se caracteriza por tener un plan muy bien definido, e ``inquebrantable'': recoger primero a todos los ni\~nos.
Puesto que ellos son los responsables de la suciedad, una vez que est\'en tranquilos se puede comenzar con la limpieza y que no sea en vano. Esta planificaci\'on para cuando el intervalo de cambio del tablero es grande, es verdaderamente ideal, pues ser\'ia 
suficiente para que el robot recoja y limpie todo antes de que se desordene el ambiente de nuevo. Pero para cuando el $t$ es peque\~no, se trata entonces de un ambiente din\'amico, en el que el objetivo de ubicar a los
ni\~nos nunca ser\'a cumplido, y por tanto el de la limpieza tampoco. 
Hemos escuchado algunas veces que el comportamiento reactivo figura en realizar acciones 
random en cada turno. 
El nuestro es aparentemente similar al proactivo, al menos en el inicio del codigo, pero en 
realidad, en lugar de enfocarse ciegamente en recoger
primero a todos ninnos siempre realiza la accion mas inmediata. Los mas inmediato realmente 
seria limpiar siempre primero, pero en ninguno de los dos
casos lo consideramos, puesto que el robot es el primero en actuar en cada turno y justo 
despues de su actuacion algun ninno puede ensuciar la casilla que
acaba de limpiar. Por eso para los dos casos tenemos que siempre que se pueda deje a un 
ninno en el corral si carga alguno y que recoja a cualquiera
que tenga cerca si no carga ninguno. Entonces, regresando a la reactividad, este ademas de 
hacer la accion mas inmediata, se mantiene recogiendo ninnos,
o al menos intentandolo, mientras el porciento de suciedad este por debajo de 40, y cuando se 
alcance este tope, ya empieza a preocuparse por priorizar la limpieza.
Tambien hemos asumido que el robot puede limpiar aunque lleve un ninno cargado. En 
resumen, creemos que un agente reactivo no es aquel que no se plantee
objetivos, claro que no, sino que es sensible al ambiente, y sabe reordenar sus prioridades 
segun como sea afectado por las condiciones en que esta.
Esta consciente de que sus planes pueden frustrarse por razones externas y por eso siempre 
hace lo que es mas seguro, m\'as urgente.

\section{Funci\'on de arranque}

Los argumentos que recibe la funci\'on principal para correr el programa son:
\begin{itemize}
\item[N:] cantidad de filas
\end{itemize}

\begin{lstlisting}
// proyecto.pl
main(N,M,Ps,Po,Ninnos,T) :-
    retractall(tablero(X)), retractall(no_ninnos(X)),
    tablero(N,M,Tablero), assert(tablero(Tablero)), assert(no_ninnos(Ninnos)),
    Mult is N*M,
    parte(Mult,Ps,Cs),
    parte(Mult,Po,Co),
    generar_tablero(N,M,Cs,Co),
    simulacion(1,T,N,M).
\end{lstlisting}


\section{Flujo de la simulaci\'on}

\begin{lstlisting}
// proyecto.pl
simulacion(Tiempo,_,N,M) :- 
    porciento_suciedad(P), 60=<P,
    mi_write(['la suciedad a alcanzado el ',P,' pociento.']),
    writeln('El robot queda despedido, termina la simulacion'),
    informe(N,M,Tiempo,'Despedido'),!.
simulacion(Tiempo,_,_,_) :-
    findall(X,sucio(X),[]), findall(X,(ninno(X),not(corral(X))),[]),
    write('todo esta limpio y ordenado, termina la simulacion'),
    informe(N,M,Tiempo,'OK'), !.
simulacion(Tiempo,Interval,_,_) :-
    Tiempo=:=100*Interval, write('se ha alcanzado 100 veces t, termina la simulacion'), 
    informe(N,M,Tiempo,'100 veces t'), !.
simulacion(Tiempo,Interval,N,M) :-
    mi_write(['\n\n\nESTAMOS EN EL MINUTO ',Tiempo]),
    robot(Robot_pos), cuadricula(Robot_pos,Cuadro),
    writeln('ACTUACION DEL ROBOT'),!,
    actua_robot_reactivo(Robot_pos,Cuadro),
    ((findall(X,  ninno(X), ListNinos), mover_Ninnos(ListNinos)) ; true),   
    Modulo is Tiempo mod Interval, try_change_enviroment(Modulo,N,M),
    T is Tiempo+1, ver_tablero,
    simulacion(T,Interval,N,M).
\end{lstlisting}


\section{Lista de predicados de Prolog m\'as utilizados}

\begin{itemize}
\item numlist/3
\end{itemize}

Ver funcionamiento de cada uno en \url{https://www.swi-prolog.org/}.

\begin{thebibliography}{5}  

\bibitem{programmingProlog}	
	Bramer, Max:
	Logic programming with Prolog,
	Springer, 
	2005
\end{thebibliography}
\end{document}